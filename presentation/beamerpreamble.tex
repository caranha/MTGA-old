%% General Beamer Presentation preable
%% With multiple options and useful Macros
%
%
% Original version by Felipe Campelo
% Changes by Claus Aranha
%
%
% Change History
% 2013.09.11
% - Added a change History

%%% DEFINED MACROS %%%%%%%%%%%%%%%%%%%%%%%%%%%%%

% Save space in lists. Use this after the opening of the list
% \compresslist

% Puts images in arbitrary sizes by specifying X and Y. 
% \Put(X,Y){\includegraphics etc etc}

% Blocks with variable horiontal sizes:
% \begin{varblock}[.8\textwidth]{blocktitle}
% \end{varblock}

%%% Packages %%%%%%%%%%%%%%%%%%%%%%%%%%%%%%%%%%%
\usepackage{amssymb,amsmath}
\usepackage{graphicx}
\usepackage{url}
\usepackage{color}
\usepackage{relsize} % For \smaller
\usepackage{url} % For \url
\usepackage{epstopdf} % Included EPS files automatically converted to PDF to include with pdflatex
\usepackage{pgfpages} % necessary for note options, npages too
\usepackage{multimedia} % for movies

\usepackage[english]{babel}
%\usepackage[latin1]{inputenc}
\usepackage{CJKutf8} % Japanese text
\usepackage{subfigure} % need to learn how to use subfigures

\usepackage{times}
\usepackage[T1]{fontenc}

%%% OPTIONS %%%%%%%%%%%%%%%%%%%%%%%%%%%%%%%%%%%%
%% Gets rid of the nav bar
\beamertemplatenavigationsymbolsempty

%%% NOTE OPTIONS %%%%%%%%%%%%%%%%%%%%%%%%%%%%%%%
% Uncomment one of these for varied note effects

%\setbeameroption{show notes} % Notes as extra slides
%\setbeameroption{show notes on second screen = left} % Notes inserted in slides
%\setbeameroption{show only notes} % Only notes

%\setbeamertemplate{note page}[plain] % Uncomment this for simpler notes


%%%%%%%%%%%%%%%%%%%%%%%%%%%%%%%%%%%%%%%%%%%%%%%%%
%For MindMaps
% \usepackage{tikz}%
% \usetikzlibrary{mindmap,trees,arrows}%

%%% Color Definitions %%%%%%%%%%%%%%%%%%%%%%%%%%%
%\definecolor{bordercol}{RGB}{40,40,40}
%\definecolor{headercol1}{RGB}{186,215,230}
%\definecolor{headercol2}{RGB}{80,80,80}
%\definecolor{headerfontcol}{RGB}{0,0,0}
%\definecolor{boxcolor}{RGB}{186,215,230}

%%% Theme Options %%%%%%%%%%%%%%%%%%%%%%%%%%%%%%%%
\mode<presentation>
{
  % A tip: pick a theme you like first, and THEN modify the color theme, and then add math content.
  % Warsaw is the theme selected by default in Beamer's installation sample files.

  %\usetheme{AnnArbor}
  %\usetheme{Antibes}
  %\usetheme{Bergen}
  %\usetheme{Berkeley}  % bem bacana - menu esquerdo
  %\usetheme{Berlin}
  %\usetheme{Boadilla}
  %\usetheme{boxes}
  \usetheme{CambridgeUS}  % bem bacana - menu superior
  %\usetheme{Copenhagen}  % not good for multiple sections
  %\usetheme{Darmstadt}  % Same as frankfurt with subsection names
  %\usetheme{default}
  %\usetheme{Dresden}
  %\usetheme{Frankfurt}  % My ``standard'' // claus
  %\usetheme{Goettingen}
  %\usetheme{Hannover}  % bem bacana - menu esquerdo
  %\usetheme{Ilmenau}
  %\usetheme{JuanLesPins}
  %\usetheme{Luebeck}
  %\usetheme{Madrid}  %bacana
  %\usetheme{Malmoe}
  %\usetheme{Marburg}  % bem bacana - menu direito
  %\usetheme{Montpellier}
  %\usetheme{PaloAlto}  % bem bacana - menu esquerdo
  %\usetheme{Pittsburgh}
  %\usetheme{Rochester}  %bacana
  %\usetheme{Singapore}
  %\usetheme{Szeged}
  %\usetheme{Warsaw}

  %%%%%%%%%%%%%%%%%%%%%%%%%%%% COLOR THEME
  %\usecolortheme{albatross}		% azul escuro, massa
  %\usecolortheme{beetle}		% cinza, menu azul
  %\usecolortheme{crane}		% branco e amarelo, massa
  %\usecolortheme{default}		% branco, azul clarinho
  \usecolortheme{dolphin}		% azul e branco, legal
  %\usecolortheme{dove}			% cinza e branco, feio
  %\usecolortheme{fly}			% todo cinza, horrível
  %\usecolortheme{lily}			% parece o default
  %\usecolortheme{orchid}		% azul e branco, ok
  %\usecolortheme{rose}			% branco e violeta-claro, bonito
  %\usecolortheme{seagull}		% cinza, feio
  %\usecolortheme{seahorse}		% nhé, meio feio
  %\usecolortheme{sidebartab}		% Azul, branco, destaque na tab, interessante
  %\usecolortheme{structure}		% bichado
  %\usecolortheme{whale}		% Azul e branco, bem bonito

  %%%%%%%%%%%%%%%%%%%%%%%%%%%% OUTER THEME
  % Theme for showing the section and sub section
  \useoutertheme{default}
  %\useoutertheme{infolines}
  %\useoutertheme{miniframes}
  %\useoutertheme{shadow}
  %\useoutertheme{sidebar}
  %\useoutertheme{smoothbars}
  %\useoutertheme{smoothtree}
  %\useoutertheme{split}
  %\useoutertheme{tree}

  %%%%%%%%%%%%%%%%%%%%%%%%%%%% INNER THEME
  %\useinnertheme{circles}
  \useinnertheme{default}
  %\useinnertheme{inmargin}
  %\useinnertheme{rectangles}
  %\useinnertheme{rounded}

  %%%%%%%%%%%%%%%%%%%%%%%%%%%%%%%%%%%
  % Changes the behavior of list uncovering, explore this // claus
  \setbeamercovered{invisible} % or whatever (possibly just delete it)
  % To change behavior of \uncover from graying out to totally
  % invisible, can change \setbeamercovered to invisible instead of
  % transparent. apparently there are also 'dynamic' modes that make
  % the amount of graying depend on how long it'll take until the
  % thing is uncovered.
}

%%% Macro for putting images at arbitrary places
%% See http://tex.stackexchange.com/questions/34921/how-to-overlap-images-in-a-beamer-slide
\def\Put(#1,#2)#3{\leavevmode\makebox(0,0){\put(#1,#2){#3}}}

%%% Variable Block
%% See http://tex.stackexchange.com/questions/12550/changing-default-width-of-blocks-in-beamer
\newenvironment<>{varblock}[2][.9\textwidth]{%
  \setlength{\textwidth}{#1}
  \begin{actionenv}#3%
    \def\insertblocktitle{#2}%
    \par%
    \usebeamertemplate{block begin}}
  {\par%
    \usebeamertemplate{block end}%
  \end{actionenv}}

%%% Save space in lists. Use this after the opening of the list %%%%%%%%%%%%%%%%
\newcommand{\compresslist}{
	\setlength{\itemsep}{1pt}
	\setlength{\parskip}{0pt}
	\setlength{\parsep}{0pt}
}
